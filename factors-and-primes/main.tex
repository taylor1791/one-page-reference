\documentclass{article}
\usepackage[landscape, margin=0.25in]{geometry}
\usepackage{multicol}
\usepackage{lmodern}
\usepackage{tabularx}

\begin{document}

\pagenumbering{gobble}

\begin{center}
\begin{huge}
  Primes and Factors
\end{huge}
\end{center}

\begin{multicols}{2}

The following tables identify if a given number less than 12,000 is prime or composite,
  and if composite, the smallest prime factor. Table 1 contains every third hundred
  beginning with zero (0, 300, 600, 900, etc); Table 2 contains every third hundred
  beginning with 100 (100, 400, 700, etc); and Table 3 contains the remaining hundreds
  (200, 500, 800, etc). The hundreds are given in the first column and the tens and ones
  on the first line at the top of each page. Numbers divisible by 2, 3, or 5 are omitted
  from the tables. Instead use the divisibility rules to the right to determine if a
  number is divisible by 2, 3, or 5.

\paragraph{Primality Test} To determine if a number is prime or composite, and if
  composite, the smallest prime factor, find the hundreds in the first column on the
  appropriate table. Then, find the tens and ones on the first line at the top of the
  table. Look on the line with the hundreds and the column with the tens and ones. If
  the space where this line and column meet contains a number, this number is the
  smallest prime factor of the given number. If the space contains two dots, ``..'', the
  given number is prime. If the necessary column is absent, the number is divisible by
  2, 3, or 5.

\paragraph{Divisibility Rule for 2} A number is divisible by 2 if its last digit is 0,
  2, 4, 6, or 8. For example, 1224 is divisible by 2 because its last digit is 4.

\paragraph{Divisibility Rule for 3} A number is divisible by 3 if the sum of its digits
  is divisible by 3. Observe that this process can be repeated until a single digit
  remains. For example, 1227 is divisible by 3 because $1+2+2+7=12$, and $1+2=3$, so 1227
  is divisible by 3.

\paragraph{Divisibility Rule for 5} A number is divisible by 5 if its last digit is 0 or
  5. For example, 1225 is divisible by 5 because its last digit is 5.

\paragraph{Prime Factorization} To find the prime factorization of a number, identify
  the smallest prime factor of the number using the divisibility rules above and the table
  below. Then, divide the number by the smallest prime factor. Repeat this process
  until the quotient is a prime number.
\end{multicols}

{\setlength{\tabcolsep}{0.5pt}
\fontsize{5}{8}\selectfont
\newcolumntype{Y}{>{\centering\arraybackslash}X}

\begin{multicols}{3}
\raggedcolumns{}
\directlua{dofile ('./tables.lua')}
\end{multicols}}

\end{document}
